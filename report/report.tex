\documentclass[11pt,a4paper,twocolumn]{article}

% --------------------------------Packages
\usepackage[margin=.75in]{geometry}
\usepackage{indentfirst}
\usepackage{titling}
\usepackage{graphicx}
\usepackage{xcolor}
\usepackage{float}
\usepackage{hyperref}
\usepackage{listings}

% -----------------------Images path setup
\graphicspath{{./images/}}

% -----------------------Code blocks setup 
\lstnewenvironment{c-darktheme}{
    \lstset{
        language=C,                        % choose the language of the code
        backgroundcolor=\color{cb-background},
        basicstyle=\footnotesize\ttfamily\color{cb-foreground}\fontfamily{Consolas},
        keywordstyle=\color{cb-keyword}\bfseries,
        commentstyle=\color{cb-comment},
        stringstyle=\color{cb-string},
        numberstyle=\color{cb-number},
        emphstyle=\color{cb-preprocessor},
        keywordstyle=\bfseries,
        numbers=left,                      % where to put the line-numbers
        numberstyle=\scriptsize\texttt,    % the size of the fonts that are used for the line-numbers
        stepnumber=1,                      % the step between two line-numbers. If it's 1 each line will be numbered
        numbersep=8pt,                     % how far the line-numbers are from the code
        float=htb,
        aboveskip=3mm,
        belowskip=3mm,
        showspaces=false,                  % show spaces adding particular underscores
        showstringspaces=false,            % underline spaces within strings
        showtabs=false,                    % show tabs within strings adding particular underscores
        tabsize=2,                         % sets default tabsize to 2 spaces
        captionpos=b,                      % sets the caption-position to bottom
        breaklines=true,                   % sets automatic line breaking
        breakatwhitespace=false,           % sets if automatic breaks should only happen at whitespace
        escapeinside={\%*}{*)},            % if you want to add a comment within your code
        morekeywords={*,var,template,new}  % if you want to add more keywords to the set
        moretypes={}                       % if you want to add more types to the set
    }
}{}

\lstnewenvironment{bash-darktheme}{
    \lstset{
        language=bash,
        aboveskip=3mm,
        belowskip=3mm,
        showstringspaces=false,
        backgroundcolor=\color{cb-background},
        basicstyle=\footnotesize\ttfamily\color{cb-foreground},
        keywordstyle=\color{cb-keyword},
        commentstyle=\color{cb-comment},
        stringstyle=\color{cb-string},
        breaklines=true,
    }
}{}

% ---------------------------Custom colors 
\definecolor{text-hl}{RGB}{190, 100, 10}
\definecolor{link}{RGB}{50, 90, 150}

% ----------------------c-darktheme colors
\definecolor{cb-background}{HTML}{1E1E1E}
\definecolor{cb-foreground}{HTML}{D4D4D4}
\definecolor{cb-comment}{HTML}{6A9955}
\definecolor{cb-keyword}{HTML}{569cd6}
\definecolor{cb-string}{HTML}{ce9178}
\definecolor{cb-number}{HTML}{b5cea8}
\definecolor{cb-preprocessor}{HTML}{569cd6}

% -----------------------------Links setup 
\hypersetup{
    colorlinks=true,
    linkcolor=link,
    filecolor=magenta,
    urlcolor=cyan,
}

% -------------------------Custom commands 
\newcommand{\hl}[1]{\textcolor{text-hl}{#1}}


%%========================================
%% Start of document
%%========================================

% \makeindex

\title{%
  Redes de computadores \\
  \large 2º trabalho prático}
\author{%
        Diogo Fernandes (202108752) \\ 
        José Sereno (202108729)}
\def\course{Licenciatura em Engenharia Informática e Computação}
\date{Dezembro 2023}

\begin{document}

% -----------------------------------Title page
\begin{titlepage}
    \begin{center}
        \includegraphics[width=0.8\linewidth]{images/uporto-feup.pdf} 
        \vspace{1cm}

        \LARGE
        \textbf{\thetitle}
        \vfill

        \large
        \textbf{\course}
        \vspace{0.5cm}

        \large
        \textbf{\theauthor}
        \vspace{0.5cm}

        \large
        \thedate
    \end{center}
\end{titlepage}
% ---------------------------------------------

\tableofcontents

\section{Introdução}

\paragraph{Objetivos}
Este projeto teve dois objetivos:
\begin{enumerate}
    \item Desenvolver uma aplicação em \hl{C} que impletmente o protocolo \hl{FTP} - descrito no \href{https://datatracker.ietf.org/doc/html/rfc959}{RFC959} - para fazer o download de um ficheiro através de um \hl{URL} - a sintaxe deste URL deveria seguir o \href{https://datatracker.ietf.org/doc/html/rfc1738}{RFC1738}.
    \item Configurar e analisar o funcionamento de uma rede de computadores.
\end{enumerate}


\section{Desenvolvimento da aplicação}

O programa \hl{download} foi desenvolvido em \hl{C} e tem como objetivo fazer o download de um ficheiro através de um \hl{URL} que segue a sintaxe do \href{https://datatracker.ietf.org/doc/html/rfc1738}{RFC1738}.
Através da realização deste programa aprofundamos os nossos conhecimentos sobre os seguintes temas:
\begin{description}
    \item[Client-Server \& TCP/IP] Aprendemos como se dá a comunicação entre um cliente e um servidor através do protocolo \hl{TCP/IP}.
    \item[RFCs] RFCs são documentos que descrevem os padrões da internet. Consultámos, como referido anteriormente, o \href{https://datatracker.ietf.org/doc/html/rfc959}{RFC959} e o \href{https://datatracker.ietf.org/doc/html/rfc1738}{RFC1738}, para além de outras documentações, para entender o funcionamento do protocolo \hl{FTP} e da sintaxe do \hl{URL}.
    \item[Sockets] Aprendemos a usar \hl{sockets} em \hl{C} como forma de comunicação entre o cliente e o servidor.
    \item[DNS (Domain Name System)] Procurámos entender o funcionamento do \hl{DNS} e como é que este é usado para traduzir um \hl{URL} num endereço \hl{IP}.
    \item[UNIX] Aprendemos alguns comandos de \hl{UNIX} que usamos no nosso programa (ex.: \hl{getaddrinfo}, \hl{socket}, \hl{connect}, \hl{recv}, \hl{send})
\end{description}

\subsection{Estrutura do código}

O código do programa \hl{download} está dividido em 3 ficheiros: \hl{download.c}, \hl{server.c} e \hl{url.h}.
É no ficheiro \hl{download.c} onde se encontra a função \hl{main} - onde se percebe o fluxo do programa. O ficheiro \hl{server.c} contém as funções que implementam o protocolo \hl{FTP} e o ficheiro \hl{url.h} contém as funções que permitem fazer o \hl{parse} do \hl{URL}.
O programa \hl{download} é compilado através do comando \hl{make} e é executado da seguinte forma:

\subsection{Fluxo do programa}

O programa é executado do seguinte comando:

\begin{bash-darktheme}
./download ftp://[<user>:<password>@]<host>/<url-path>
\end{bash-darktheme}


\section{Configurar e analisar o funcionamento de uma rede}

O objetivo deste conjunto de experiências é configurar uma rede de computadores de modo a que estes tenham acesso à internet para instalar ficheiros a partir de um servidor remoto usando o protocolo FTP desenvolvido ou seja, a aplicação.


\subsection{Exp 1- Configurar o IP de uma rede}

O objetivo desta experiência é conectar o Tux-3 ao Tux-4 a partir do MicroTik Switch.

\end{document}

